\section{Literature Review}
My paper relates to two strands of literature on central bank communication: i) its impact on financial markets, and ii) its quantification through textual analysis. The academic literature investigating the impact of the ECB’s communication and especially speeches on financial markets is scarce. For this reason, to have sufficient evidence in constructing hypotheses I also include papers exploring communication of other central banks where needed.

\subsection{Central Bank Communication and Asset Prices}
There is substantial evidence that central bank communication moves asset prices, suggesting that the central bank communication conveys relevant information to market participants, i.e., it contains news. For example, see the evidence on interest rates for the ECB (\cite{ehrmann2007, schmeling2019, rosa2007,brand2010,rosa2011,lamla2011}), the FED (\cite{kohn2003,gurkaynak2004,lucca2009}), the Bank of England (\cite{reeves2007}), the central banks of Czech Republic, Hungary and Poland (\cite{rozkrut2007}), the central bank of New Zealand (\cite{guthrie2000}) and the Swedish central bank (\cite{andersson2006}). Also, several papers discovered that central banks’ communication can steer the interest rates in the intended direction, implying that the communication is a monetary policy tool that allows the central banks to manage the expectations of the market participants. For instance, \textcite{guthrie2000} found that following the Reserve Bank’s of New Zealand tightening announcements, interest rates of all maturities increased. They showed that this increase is not caused by the simultaneous changes in the open market operations. In the same vein, \textcite{rosa2007} investigated the ECB press conferences and discovered that when the ECB’s communication is dovish (hawkish), the interest rates decrease (increase). Some studies even showed that central bank communication has a more significant impact on the bond market than the changes in the policy rate (see, e.g., \cite{gurkaynak2004,rosa2011}). \textcite{gurkaynak2004} explained that the policy rate changes in the US and Eurozone are rarely a surprise. For this reason, the tone used to explain the monetary policy decisions is the primary driver of the financial markets because the tone reveals the information regarding the path of the future policy expectations. Precisely, through this channel, the central banks guide the private sector's expectations.

Most of the literature on central bank communication analyzed press conferences that are used to convey monetary policy decisions and the reasons behind them. The speeches of central banks, on the other hand, have been overlooked. \textcite{kohn2003} are the first that studied the impact of speeches on interest rates. They investigated the speeches of FED that took place between 1989 and 2003 but did not find any influence on the interest rates. The researchers argue that speeches are too noisy, as they are given about a broad spectrum of topics, and some may have little to do with monetary policy or the economic situation. Nevertheless, more recent papers investigating the speeches of the ECB and other central banks found a significant impact on interest rates (\cite{andersson2006,rozkrut2007,ehrmann2007}). Furthermore, \textcite{reeves2007} and \textcite{connolly2004} found somewhat significant results. Across the total sample, they found very little influence on the interest rates. However, several speeches appear to have a significant impact on the volatility of the interest rates. As \textcite{kohn2003} they argue that the inclusion of all the speeches in the dataset hides the effect of speeches that are deliberately designed to influence expectations. For this reason, I intend to augment this approach by performing a pre-selective filter based on keyword frequency per speech to exclude the less relevant speeches. In doing so, I hope to minimize the noise experienced by the researchers above. Moreover, speeches, among other communication tools of the ECB, have become more critical for the conduct of monetary policy in the current low-inflation environment (\cite{bernoth2020}). Therefore, based on the above mentioned arguments, I hypothesize that:

\textbf{Hypothesis 1a:} The tone of the ECB speeches has an impact on interest rates.\\
\textbf{Hypothesis 1b:} When the tone of the ECB speeches is dovish (hawkish) the interest rates decrease (increase).


Delving deeper into the ECB’s communication, the impact of speeches on different maturities of interest rates is a natural area to investigate next. The ECB’s communication has a more substantial impact on the interest rates with the shorter-term maturities. For example, \textcite{lamla2011} found that the tone in the ECB’s press conferences only affects the interest rates of 4-12 months. They argue that given a communication signal today, financial markets expect the ECB to alter interest rates the soonest four months later, but not earlier. This indicates that the ECB prepares the market well in advance for the changes in the policy rates. Similarly, \textcite{musard-gies2006} and \textcite{rosa2007} discovered that the short end of the yield curve specifically yields with maturities of 6 and 12 months, react the most to the tone of the ECB’s press conferences. On the other hand, \textcite{ehrmann2007} found that the tone of the ECB’s speeches affects the interest rates with maturities even up to five years. It seems intuitive that the ECB’s communication mainly impacts short-term interest rates. The ECB cannot provide meaningful policy intentions more than several years ahead because future monetary policy decisions depend on the future economic environment. Also, the longer the maturity of interest rates, the less control the ECB can exercise because longer-term interest rates depict exogenous factors such as global interest rate trends and changing term premia (\cite{lucca2009}). This leads to my following hypothesis to be tested:


\textbf{Hypothesis 1c:} The tone in the ECB’s speeches has a stronger impact on the short end of the yield curve than on the long end.


During the press conference meeting of 12 March 2020, the ECB President Christine Lagarde mentioned that it is not the ECB’s role to “close the spread” referring to the gap between Italian and German bond yields. Immediately after her statement, Italy’s bond yields rose from 117bps to 174 bps, whereas Germany’s bonds remained at -74bps, indicating a 35\% increase in the spread (\cite{leombroni2021}). It was the biggest single-day increase in Italian government bonds in a decade. The market interpreted that the ECB will no longer be the lender of the last resort to Italy, which triggered Italy’s bond market sell-off. Lagarde immediately corrected this perception and apologized for her “botched” communication (\cite{arnold2020}). Recent academic literature explored the role of the ECB’s communication on yield spreads. For example, \textcite{schmeling2019} found that the yield differential of BBB- and AAA-rated corporate bonds decreases (increases) when the tone in the press conferences of the ECB becomes more positive (negative). The researchers argue that the tone of the central bank communication influences the risk aversion of the market participants. For this reason, the default spread changes. \textcite{beaupain2020} also investigated the impact of the press conferences of the ECB on the yield spread. They provide supporting evidence that the ECB’s communication complements its actions in reducing the yield spread between the ten euro countries and Germany. The ECB’s communication was especially effective in reducing the yield spread for “distressed” countries such as Italy, Spain, Greece, Portugal, and Ireland. Following this, I am curious to explore whether the spread between Italian and German sovereign bonds changes due to sentiment in speeches. Therefore, based on the arguments above, my next hypotheses are accordingly:


\textbf{Hypothesis 2a:} The tone in the ECB’s speeches impacts the spread between Italian and German sovereign bonds.\\
\textbf{Hypothesis 2b:} When the tone is dovish (hawkish) the spread between Italian and German sovereign bonds decreases (increases).

In 2005 former ECB president Jeane-Claude Trichet in his speech, said, “I will argue that under some conditions the central bank can regain control of private expectations without necessarily changing interest rates, but by being visibly and credibly “alert”, explaining and stressing its commitment to maintaining inflation at levels consistent with the price stability objective” (\cite{europeancentralbank2005}). Unfortunately, very little research is done on the effects of the ECB’s communication on inflation expectations to support his statement. \textcite{ehrmann2007} are the first to investigate whether the inflation expectations react to the tone in the ECB’s speeches. However, they do not find any statistically significant results. The findings of \textcite{jansen2007}, to some extent, support Trichet’s statement. They found a significant relationship between the tone in press conferences and changes in break-even inflation but only during October and November 2005. The researchers argued that communication works in steering inflation expectations when the central bank credibly commits to a particular inflation goal, i.e. when the deeds support the communication. \textcite{ullrich2008} also provides somewhat supporting evidence to Trichet’s statement but again stresses the importance of credible commitment. She investigated the influence of the tone in the ECB’s press conferences on the inflation expectations of experts. She found that a hawkish tone, i.e., addressing the concern about inflation risks, induced financial market experts to adjust inflation expectations with a 6-month horizon upwards. This is caused by the experts’ disbelief in the ECB’s ability to act against this rising inflation in the upcoming six months. However, inflation expectations eventually adjust downward, as shown by \textcite{hubert2017}. He found that a hawkish tone in the ECB’s speeches has a negative effect on the long-horizon inflation expectations, but only if they are considered alone, without the ECB’s inflation projections. Following this, I aim to expand this scarce literature on inflation expectations by investigating the tone of the ECB’s speeches. I do not expect to find a strong relationship since the tone only indirectly impacts inflation expectations. My study is closest to \textcite{ehrmann2007}; however, given that a more recent study by \textcite{hubert2017} found somewhat significant results, my hypotheses are as follows:


\textbf{Hypothesis 3a:} The tone in the ECB’s speeches impacts inflation expectations.\\
\textbf{Hypothesis 3b:} When the tone is dovish (hawkish) there is an upward (downward) change in inflation expectations.


Besides the ECB’s communication impact on interest rates and inflation expectations, some studies also explored the relationship between the ECB’s communication and stock markets. For example, \textcite{ehrmann2007} found that when the ECB’s speeches carried tightening (loosening), monetary policy sentiment the Euro Stoxx 50 index decreased (increased). This is consistent with \textcite{schmeling2019} who investigated the tone in the press conferences of the ECB. \textcite{picault2017} also found that the Euro Stoxx 50 index reacts to the tone in the ECB’s press conferences. In addition, \textcite{apergis2019} explored the impact of ECB’s announcement wire news on the major international stock indexes, i.e., S\&P500, FTS100, CAC40, DAX, and Nikkei225. Specifically, when the tone in the ECB’s messages was positive, the stock returns were positive, whereas the volatility was negative and vice versa. \textcite{schmeling2019} explained that the positive sentiment surrounding monetary policy reduces peoples’ risk aversion resulting in higher willingness to invest in equity markets. For this reason, the stock prices increase. Following this I also aim to explore the relationship between the tone in the speeches and the stock market returns. Based on the evidence above I hypothesize:


\textbf{Hypothesis 4a:} The tone in the ECB’s speeches impact the stock returns.\\
\textbf{Hypothesis 4b:}  When the tone is  dovish (hawkish) the stock returns increase (decrease).

Thus far I looked at studies investigating the sentiment surrounding the monetary policy talk. However, a large portion of the ECB’s communication is also dedicated to provide the economic outlook prospects in the Eurozone. \textcite{ehrmann2007} is the only study that investigated the economic outlook tone of the ECB’s speeches. They discovered significant relationship between the economic outlook sentiment and financial assets such as Euro Stoxx 50 and short-term interest rates. However, no impact was found on inflation expectations. Similarly, \textcite{picault2017} explored the economic outlook tone embedded in ECB’s press conferences and found a significant impact on the stock market. Both papers found that a positive economic outlook tone led to an improvement in asset prices. I intend to expand this scarce literature on the economic outlook tone for the ECB’s speeches. Based on the evidence above, my hypotheses are accordingly:

\textbf{Hypothesis 5a:} The tone surrounding economic outlook in the ECB’s speeches matters for interest rates, credit spread, stock market, but not inflation expectations.\\
\textbf{Hypothesis 5b:} The positive (negative) economic outlook tone in the ECB’s speeches leads to an increase (decrease) in stock returns and decrease (increase) in interest rates and credit spread.


\textcite{ehrmann2007} assessed the communication strategies of the Fed and the ECB. They found that the US financial markets respond significantly more strongly to communication by Greenspan and less to statements by other FOMC members whereas Euro area markets react similarly to communication by the President and other Governing Council members. Unlike in the FED, the decision making in the ECB is collegial. For this reason, financial market agents react the same to the communication by the policy makers of different ranks in the Euro area. My obtained speech dataset is sorted into speeches by President, Vice-President and the remaining Board Members. Therefore, I also aim to explore whether my results support the findings of \textcite{ehrmann2007}. Given the evidence above, I hypothesize:


\textbf{Hypothesis 6:} Financial markets respond the same to the speeches by President, Vice-President and Board members.

%%%%%%%%%%%%%%%%%%%%%%%%%%%%%%%%%%%%%%%%%%%%%%%%%%%%%%%%%%%%%%%%%%%%%


\subsection{Quantifying Central Bank Communication}
Two main methods have been used in literature to quantify qualitative information. One is \textit{manual content analysis}, which involves reading a text, classifying it based on its content, and then coding it on a numerical scale. For example, \textcite{musard-gies2006} and \textcite{rosa2007} hand-coded the ECB’s press conferences based on the tone of the content into a grid-scale from -2 (very dovish) to +2 (very hawkish). Although manual classification is easy to carry out, it is labor-intensive and subjective. For instance, Rosa and Verga (in \cite{rosa2007}) disagreed on 22.58\% of classified statements.


More recent literature considered the drawbacks of manual content analysis and instead used an \textit{automated approach}. Specifically, a computer program is used to count how frequently certain words appear based on a pre-specified word list or dictionary. In dictionaries, words are allocated under specific topics and sentiments. For example, “unemployment” would be considered a negative word under the dictionary of \textcite{loughran2011}. If a text contains a lot of negative words like “unemployment” (other examples “uncertainty”, “crisis”, etc.) the text will likely convey a pessimistic message. Therefore, a central assumption is that the frequency of the words reflects the tone of the text. The score is then computed for example as the number of positive words minus the number of negative words scaled with total words in the text (see e.g., \cite{picault2017,baranowski2021}. The benefits of dictionaries are two-fold, convenience and robustness. Convenient because prebuilt lexicons are developed by NLP experts with certain classifications and/or types of source data in mind. If available, they are easily applied to one’s dataset and require relatively little programming for classification. Robust because the developers of most lexicons attempt to develop their lexicon so that it applies across as many datasets as possible without losing the predictive power of classifications.


\textcite{schmeling2019}, in their study, used \textcite{loughran2011}’s (LM) financial dictionary to quantify the press conferences of the ECB. They verify that the dictionary indeed captured how the ECB frames its macroeconomic fundamentals. \textcite{baranowski2021} also identified the LM dictionary as appropriate to measure the ECB’s monetary policy stance. However, \textcite{picault2017} are skeptical about the LM dictionary as it is a single-word dictionary and is not precisely designed for the ECB’s talk. The LM dictionary does not capture the monetary policy or the economic outlook sentiment specifically. It simply assigns words into either having positive or negative sentiment. Therefore, it may fail to capture the true ECB’s tone. However, if my speeches are too noisy to extract monetary policy or economic outlook sentiment as argued by \textcite{kohn2003}, then the LM dictionary is more appropriate as it derives a general tone of the communication – positive or negative. Given the discussion above, I intend to apply the LM dictionary to my speech dataset.


\textcite{baranowski2021} also identified \textcite{bennani2017}’s (BN) dictionary as another potential dictionary that correctly depicted the tone of the ECB’s communication. The BN dictionary is developed from the speeches of the members of the ECB’s Governing Council; therefore, the dictionary may closely reflect the context of my speech data set. However, the ECB’s speeches that I investigate are given only by the Executive Board members. Hence, there may be a mismatch in the structure and style of the communication between the BN dictionary and my speeches dataset. Also, the BN dictionary is a single-word dictionary. Therefore, there is a high chance of misclassifying certain words as it does not consider the context. Despite this, I believe the BN dictionary may be able to depict the tone in the ECB’s speeches to some extent. Therefore, I intend to apply it for robustness.


Finally, the most suitable to my speeches’ data is \textcite{picault2017}’s (PR) dictionary. The dictionary is constructed from the ECB’s press conferences and is divided into monetary policy and the economic outlook topics. In addition, the lexicon consists of a sequence of words, from 1-gram to 10-gram, where gram refers to a word. PR lexicon is likely to have only a small share of misclassifications because a sequence of 10 words can greatly capture the context of the text, at least much better than a single- or two-word dictionaries identified above. \textcite{picault2017} showed that its dictionary outperformed all of the above identified  dictionaries. However, the potential incompatibility between the PR-lexicon’s source data and the ECB's speech dataset used in this paper should not go unnoticed. Two possible limitations come to my mind, different types of speeches and different speakers. First, the PR-lexicon is based on the often structured and standardized ECB monetary policy decisions press conferences. In contrast, this paper’s dataset contains a broader array of speeches, some of which have little to do with monetary policy or economic outlook. Second, as the PR-lexicon was developed over the 2006-2014 timespan, the vocabulary was based on a smaller set of speakers. For example, given the switch in ECB presidents from Mario Draghi to Christine Lagarde, potential nuances in their different speech styles may be lost using the PR-lexicon. Furthermore, my dataset includes speeches given by not only the ECB president but also by its vice president and executive board members. Despite these mismatches, the PR tone is likely to capture the tone well in my speeches’ data set.


Given that the PR dictionary is the most compatible with my speech data set, I will use it as my leading dictionary. The remaining two - the BN and the LM are for robustness. In addition, I am the first to apply all the three identified dictionaries to the ECB’s speech data set. 

