\section{ECB's Communication Overview}

\subsection{Why ECB's Communication Matters?}
The primary objective of the ECB is to maintain price stability. The main tool to achieve this is by manipulating short-term overnight interest rates. However, the ECB understands that market agents do not make decisions based on these short-term interest rates. Instead, what matters is the ECB's signals to the market via alteration of these rates. These signals affect market expectations regarding the future path of short-term interest rates, which then impact long-term interest rates as shown by expectations hypothesis of the term structure of interest rates in Equation \ref{eq:expect} (\cite{blinder2008}).

\begin{equation}
    \label{eq:expect}
    R_{t} = \alpha_{n} + (1/n)(r_{t} + r_{t+1}^{e} + r_{t+2}^{e} + \dots + r_{t+n-1}^{e}) + \varepsilon_{1,t}
\end{equation}

\vspace{-16pt}

where $R_{t}$ is long-term interest rate, $r_{t}$ is current short-term interest rate of $n-1$ days in the future and $\alpha_{n}$ is the term premium, assumed to be stochastic ($\varepsilon_{t}$). Long-term interest rates matter because they influence private agents' investment decisions, which drive the economy. 

Since the financial crisis, the overnight rate is stuck near zero. Therefore the ECB introduced unconventional monetary policy tools and intensified its communication efforts to deliver its signals. With its communication, the ECB sends signals about the future short-term interest rates, and through these rates, it can influence the long-term rates as seen in Equation \ref{eq:expect}. This form of communication is called "forward guidance".
 

If agents are rational and are fully aware of the economic conditions and the ECB's reactions to them, the ECB's communication would be redundant (\cite{woodford2005}). Agents would not react to the signals of the ECB as they already incorporated them into their expectations once the new information came out. However, in reality ECB's communication matters for several reasons. First, the ECB updates its economic model. It incorporates new parameters, updates its policy strategy, etc.,  which take time for the market agents to learn about. Second, agents sometimes may act irrational. Third, the ECB specialises in analysing the economy, therefore it may have more and better knowledge. In addition, the ECB's communication becomes especially important during recessions as crises can make it difficult for the market agents to interpret the information and anticipate monetary policy responses (\cite{coenen2017}). For all these reasons above it is evident that the ECB's communication has an influence on agents behaviour and hence the economy. 

Recognizing this influence, the ECB continuously aims to improve how it conducts its communication as it is of the utmost importance that the market participants correctly understand the ECB's signals. The ECB notes on its website: "how people spend and invest today, and expect the economy to develop tomorrow, can affect our job of keeping prices stable" (\cite{europeancentralbank2021}). Therefore, the ECB continuously simplifies its communication, stresses its commitment to what it says, and increases its communication frequency to help people better understand its decisions and build trust. Today the ECB uses a large number of tools to communicate with the public. These are press releases, press conferences, the blog, monetary policy accounts, speeches, interviews, and listening events. 
%%%%%%%%%%%%%%%%%%%%%%%%%%%%%%%%%%%%%%%%%%%%%%%%%%%%

\subsection{The Informativeness in the Speeches vs. Other Means of the ECB's Communication}


One of the ECB's communication means that gained recent attention is speeches. The number of speeches over the years has been gradually increasing (see Figure \ref{fig:figcombo1}). The ECB uses speeches to explain its monetary policy decisions and address other topics related to central banking, the financial system, and the economy. Therefore, speeches do not necessarily contain new information but instead clarify the information delivered through other communication means, mainly press conferences. Press conferences provide the public with updated monetary policy decisions and economic outlook projections. Therefore, one may think of speeches as a complement to the press conferences. 

Unlike press conferences which are held every six weeks immediately after new monetary policy decisions, speeches are flexible in their timing. Also, in comparison to press conferences that represent the Governing Council's view as a whole, speeches are given by the members of the Executive Board and represent individual views of the speakers. Moreover, press conferences have a standardized format and language, whereas speeches do not follow strict guidelines. Other communication tools such as interviews, the blog, and listening events are primarily used for public relations; therefore, I will not address them further.  

The role of the press conferences as a monetary policy tool is well understood in practice and the academic literature. However, not many researchers have spent time analyzing speeches, given that they are significantly less structured than the monetary policy press conferences. Regardless, provided that the ECB's representatives go through the effort of giving these speeches, one expects them to align with the bank's overall objective - price stability through manipulating the interest rates. Therefore, it is interesting whether speeches contain unique signals to the market agents and hence can be declared as another monetary policy tool to manage market expectations like press conferences or is it just a public relations tool like the interviews. I test this empirically.  

